%% Para compilar o programa, use pdflatex
%% Opcoes:
%%
%% Cursos:
%% cic: Bacharelado em Ciência da Computação
%% tsi: Técnologo em Sistemas para Internet
\documentclass[cic]{ifbclass/ifbclass}

\address{BRASÍLIA}

\title{Título do Trabalho}

\date{2018}

\author{Nome completo do Autor}
\adviser{Nome completo do Orientador}
\coadviser{Nome dompleto do co-orientador }

% Macros (Se necessario, defina suas proprias aqui)
\def\x{\checkmark}
%\let\lstlistoflistings\origlstoflistings
  \begin{document}

\frontmatter

\frontpage

\presentationpage

% Quando a biblioteca preparar a ficha catalografica, insira-a aqui
% Antes da defesa do trabalho, crie uma ficha falsa
\begin{fichacatalografica}
  \FakeFichaCatalografica
%     \includepdf{fig_ficha_catalografica.pdf} % crie o arquivo come esse nome e descomente essa linha
\end{fichacatalografica}

\banca

\begin{dedicatory}
I dedicate this thesis to all my family, friends and professors who gave me the
necessary support to get here.
\end{dedicatory}
  
\acknowledgements
Agradeço ao meu orientador Prof. Dr. Nome do Orientador, pela sabedoria com que me guiou nesta trajetória.

Aos meus colegas de sala.

A Secretaria do Curso, pela cooperação.

Gostaria de deixar registrado também, o meu reconhecimento à minha família, pois acredito que sem o apoio deles seria muito difícil vencer esse desafio. 

Enfim, a todos os que por algum motivo contribuíram para a realização desta pesquisa.


\begin{epigraph}[]{Márcio de Deus}
When one finds a hard problem, the more complicated it is, the more one ought to work towards enlightening it's solution.
\end{epigraph}

\resumo
% Escreva seu resumo no arquivo resumo.tex
{\parindent0pt
  SOBRENOME, Prenome do Autor do Trabalho. Título do trabalho: subtítulo (se houver).  2018. 65 f. 
Trabalho de Conclusão de Curso (Graduação) – Tecnólogo em Sistemas para Internet. 
Instituto Federal de Brasília – Campus Brasília. Brasília/DF, 2018.
\vspace{1cm}

Elemento obrigatório, constituído de uma sequência de frases concisas e objetivas,
fornecendo uma visão rápida e clara do conteúdo do estudo. O texto deverá conter no
máximo 500 palavras e ser antecedido pela referência do estudo, com exceção do resumo
inserido no próprio documento. Também, não deve conter citações. O resumo deve ser redigido
em parágrafo único, espaçamento simples e seguido das palavras representativas do conteúdo
do estudo, isto é, palavras-chave, em número de três a cinco, separadas entre si por ponto e
finalizadas também por ponto. Usar o verbo na terceira pessoa do singular, com linguagem
impessoal (pronome SE), bem como fazer uso, preferencialmente, da voz ativa.

\begin{keywords}
Primeira palavra. Segunda palavra. Terceira palavra. Quarta palavra. Quinta-palavra.
\end{keywords}

}
  
\abstract
% Escreva seu abstract em ingles no arquivo abstract.tex
{\parindent0pt
  SOBRENOME, Prenome do Autor do Trabalho. Título do trabalho: subtítulo (se houver).  2018. 65 f. 
Trabalho de Conclusão de Curso (Graduação) – Tecnólogo em Sistemas para Internet. 
Instituto Federal de Brasília – Campus Brasília. Brasília/DF, 2018.
\vspace{1cm}

Elemento obrigatório, constituído de uma sequência de frases concisas e objetivas,
fornecendo uma visão rápida e clara do conteúdo do estudo. O texto deverá conter no
máximo 500 palavras e ser antecedido pela referência do estudo, com exceção do resumo
inserido no próprio documento. Também, não deve conter citações. O resumo deve ser redigido
em parágrafo único, espaçamento simples e seguido das palavras representativas do conteúdo
do estudo, isto é, palavras-chave, em número de três a cinco, separadas entre si por ponto e
finalizadas também por ponto. Usar o verbo na terceira pessoa do singular, com linguagem
impessoal (pronome SE), bem como fazer uso, preferencialmente, da voz ativa.

\begin{keywords}
Keyword. Second keyword. Third keyword. Keyword.
\end{keywords}

}

% Lista de figuras
\listoffigures

% List of Codes
\lstlistoflistings

% Lista de tabelas
\listoftables

% Lista de acronimos
% Acronyms manual: http://linorg.usp.br/CTAN/macros/latex/contrib/acronym/acronym.pdf
\listofacronyms
\input{text/acronyms}

% Sumario
\tableofcontents

\mainmatter

\include{text/introduction}
\include{text/background}
\include{text/desenv}
\include{text/conclusion}

% Referencias

\begin{references}
  \bibliography{bib/references}
\end{references}

% Apendice

\theappendix
\include{appendix/mapping-study}

\end{document}
